\begin{abstract}
This article is part of the\textit{ STIC-B415 (Information Systems Architecture)} course. And the choice of this subject stems not only from my interest in artificial intelligence, but also from my initial concern to find out more about deepfake technologies and their role in the movement of false information. The aim of this article is not to go into the theoretical depths of the notions of deepFake and disinformation, but to analyze how the existence of deepFakes feeds disinformation. DeepFakes are the result of deep machine learning (as we'll see later), which in turn is a branch of the vast field of artificial intelligence (\emph{a process of imitating human intelligence, based on the creation and application of algorithms executed in a dynamic computer environment. Its aim is to enable computers to think and act like human beings \cite{IntelligenceArtificielleDefinition}).}In this article, we'll look at a few theoretical concepts that will help us to better understand how deepFakes work. We will then relate deepFakes to disinformation, to understand how this in itself constitutes a vector for the propagation of false information. We'll also discuss possible solutions that can help reduce the rate of disinformation linked to deepFakes and, finally, we'll comment briefly on a particular case of the use of deepFakes to misinform the public.
\end{abstract}