\section{Conclusion}
To conclude our article, we saw that by examining the importance of artificial intelligence in the modern disinformation process, deepfake makes us to stand at the crossroads between technological advances and societal problems. The use of artificial intelligence as an emancipatory tool has also given rise to moral questions and hidden threats, as deepfake demonstrates.

To better understand how deepfake works and to expose the subtle mechanisms of disinformation, our study examined the theoretical underpinnings. We examined the effects of this alliance between AI and disinformation, and looked at how it affects our institutions and society.There are many obstacles, but despite the threats, there are also opportunities for progress. Solutions come from a deeper understanding of the issues, responsibility and ethics involved.
The digital landscape must be regulated by enlightened regulations, and responsibility for the use of AI must be scrupulously scrutinized. Our collective ability to anticipate, educate and innovate is essential to society's ability to resist misinformation fueled by artificial intelligence. A digital future that values trust, transparency and truth requires a balance between technological advances and ethical safeguards. We look to the future, calling for informed vigilance and resolute action to shape a digital age where artificial intelligence serves truth, not distortion.
