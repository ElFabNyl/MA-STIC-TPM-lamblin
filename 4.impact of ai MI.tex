\section{ DeepFake and misinformation}
So far in our article, we've taken a look at artificial intelligence and explored the notion of deepfake, which uses deep learning techniques to generate models that tend to substitute for reality. Aside from all the good this technology can do, particularly in the film industry to create certain comic tricks, deepfake, in the wrong hands, serves more as a disinformation engine than anything else. 
In fact, to be informed in the past, before the explosion of the Internet, people had to go through the traditional media of the time, such as paper newspapers, radio and television. The institutions in charge of these media had a firm grip on them, which meant that the information published was verified and the sources were reliable. 
Today, however, \emph{there's a global pandemic, thanks to the information circulating on television, your tablet or your phone. The Internet makes information available anywhere, at any time, in just two clicks \cite{laetitiaCommentInformaitonAvant2021}.} With the Internet, people have access to a huge amount of information from a variety of sources. And this abundance of information makes it extremely difficult to distinguish the true from the false. 
\emph{Information is a decision-making tool, or even a means of influencing reality \cite{boydensOceanDonneesCanal2012a}.} If information is what allows an individual to make a decision, and with the advent of the internet and the impossibility of controlling what individuals publish as information on their networks, also considering that a significant portion of the global population no longer has the habit of obtaining information from reputable media dedicated to this purpose, then deepfakes, in themselves, contribute more to misinformation than anything else. This is because anyone can have access to create or alter existing information and publish it.

DeepFakes are often used to create falsified speeches by public figures, as we'll see in a case study. It can also generate fake videos of historical events...etc. And one of the immediate consequences of this is that individuals may be reluctant to believe when confronted with real information, which slows down their decision-making considerably.
