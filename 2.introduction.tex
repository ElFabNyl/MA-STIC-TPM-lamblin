\section{Introduction}
Artificial intelligence (AI) is growing very fast, and has had a significant impact on various aspects of our society, offering unprecedented opportunities while raising significant issues. One of this issue is misinformation. It has become one of the major problems of the modern world, given that to understand the world in which we find ourselves, and to make decisions in most cases, we need to be informed. We need data on our environment. The spread of misinformation has been revolutionized by AI's ability to create synthetic content, thanks in particular to deepfake technology. What's even more worrying is that \emph{Social network algorithms, far from being neutral, are not designed to sort the true from the false, but to select, classify, prioritize and target information likely to capture the attention of a maximum number of users\cite{InternetAutorouteDesinformation}.} This means that social networks frequently focus on user retention and advertising revenue generation. And as a result, algorithms are designed to favor content that provokes reactions, whether positive or negative. This dynamic creates an information bubble where users are often exposed to content that reinforces their current preferences, without necessarily evaluating the veracity of the information presented. As AI becomes more and more established in various sectors, disinformation finds new ways to spread. It is therefore necessary to pay particular attention, from understanding the mechanisms of AI to the very nature of misinformation and the theoretical underpinnings of this complex issue. The first objective of this article will be to establish the theoretical foundations of our analysis, define artificial intelligence clearly and examine its applications. In addition, we will explore the complex mechanisms of misinformation, which will help us to better understand the role of AI in this context, particularly through deepfake.
The second section will examine the real effects of AI, highlighting the speed with which misinformation spreads, its destabilizing effects on public trust and its far-reaching consequences for institutions and society as a whole.
The fourth section will identify the challenges and issues inherent in this complex reality, focusing on detecting and combating deepfakes, as well as the crucial issues of responsibility, ethics and regulation.
The fifth section will present solutions and perspectives for reducing the negative effects of AI on misinformation. Overcoming this global challenge will require the development of innovative detection technologies, public awareness initiatives and working together on an international scale.
Finaly In part six, we will illustrate our analysis with concrete case studies, examining an example of deepfakes used for disinformation and analyzing their effects on society and individuals.
